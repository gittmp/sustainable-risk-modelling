% ------------------ DOCUMENT SETUP / PACKAGES ------------------ 
\documentclass[a4paper, 12pt]{report}
\usepackage[a4 paper, top=25mm, bottom=25mm]{geometry}
% \usepackage{times}

\usepackage{float}
\usepackage{color}
\usepackage{amsmath}
\usepackage{emptypage}
\usepackage{array}

\usepackage{graphicx}
\graphicspath{{./resources/}} 

\usepackage{setspace}
\setstretch{1.0}

% Manages hyperlinks 
\usepackage[colorlinks=true, linkcolor=black, urlcolor=red]{hyperref}

% Bibliography package
\usepackage[round]{natbib}
\bibliographystyle{plainnat}

% Line Break Properties
% \tolerance=1
\emergencystretch=\maxdimen
% \hyphenpenalty=10000
\hbadness=10000



% This package provides an easy way to input latin sample text (for the template only)
\usepackage{blindtext}



% Title page information
\title{Energy-Efficient Deep Learning for Finance}
\author{Tom Maxwell Potter}
\date{\today}


% ---------------------  DOCUMENT ----------------------
\begin{document}

    % ------------------  TITLE PAGE -------------------
    \begin{titlepage}
        \begin{center}
            % UCL Image
            \vspace*{1cm}
            \makebox[\textwidth]{\includegraphics[width=.5\paperwidth]{resources/UCL_LOGO.png}}
            
            \vfill
            
            % Title
            \makeatletter
            {\Huge\textbf{\@title}}

            % \vspace{0.8cm}
            % \linebreak
            % {\large\textbf{Scientific Computing Individual Research Project}}

            \vspace{0.8cm}
            by
            \vspace{0.8cm}

            % Author
            {\Large\textbf{\@author}}

            % Date
            \vspace{1.5cm}
            {\textbf{\\\@date}}

            \vfill

            {A dissertation submitted in part fulfilment\\
            of the requirements for the degree of\\}
            {\setstretch{2.0}
            \textbf{Master of Science}\\
            of\\
            \textbf{University College London\\}}
            \vspace{1cm}
            {Scientific and Data Intensive Computing\\
            Department of Physics and Astronomy}

            \vspace{2cm}
        \end{center}
    \end{titlepage}


    % -----------------  DECLARATION  -------------------
    \pagenumbering{roman}
    \chapter*{Declaration}
    \addcontentsline{toc}{chapter}{Declaration}
    
    I, Tom Maxwell Potter, confirm that the work presented in this thesis is my own. Where information has been derived from other sources, I confirm that this has been indicated in the dissertation.


    % ----------------------  ABSTRACT -----------------------
    \newpage
    \addcontentsline{toc}{chapter}{Abstract}

    \begin{abstract}

        This thesis investigates the use of energy-efficient methods for deep learning-based financial volatility forecasting, aiming to reduce the energetic cost of such models and demonstrate how the sustainability of deep learning for finance can be improved.

        \textbf{Context/background:} The financial sector has long been associated with largely negative environmental, social, and governance (ESG) impacts, including being a major contributor to global carbon emissions. Despite the attempts by some to prioritise \emph{sustainable finance}, the recent expansion of financial technology---incorporating new, expensive methods such as \emph{deep learning} (DL)---has only worsened the energy consumption attributed to this industry, accelerating its carbon emissions.
        
        \textbf{Aims:} In an attempt to address these negative impacts of financial technology, this project aims to develop an energy-efficient DL system for financial modelling. This research will explore \emph{Green AI} methods that attempt to reduce the energy expended training DL models, and apply these for the first time to models used in finance. To exemplify the benefits of these methods, a performant financial volatility model will be developed that not only produces accurate results, but prioritises generating this performance in an efficient manner, minimising the energy and data resources required during training. This system aims to demonstrate that the principles of Green AI are applicable within the financial sector, furthering the scope of sustainable finance by improving the sustainability of deep learning for finance and, hence, minimising the ESG impacts of the financial sector.
        
        \textbf{Method:} This research will commence with an analysis of the resource requirements of typical systems in the field of deep learning for finance. A particular focus will be given to the domain of financial volatility modelling, as this is a major application of deep learning in finance, and the \emph{long short-term memory} (LSTM) networks typically exploited for such tasks. Energy and data-efficient training methods will be explored, developing a deep model that consumes less energy and requires less data to train, but maintains accurate performance. Specifically, methods such as \emph{active learning}, \emph{progressive training}, and \emph{mixed precision} will be explored that reduce the resource requirements of training, proving the feasibility of efficient models within this field.
        \\ \\ \\
        \textbf{Outline of research:} 
        \begin{itemize}
            \item \underline{Chapter \ref{section: baseline}: \emph{Baseline finanical volatility model}}. An initial deep model will be implemented, using a traditional training process and LSTM architecture, to act as an exemplar of the resource requirements of this domain.

            \item \underline{Chapter \ref{section: energy-extensions}: \emph{Energy-efficient training extensions}}. Several adaptations to the model training process will be made that prioritise reducing the energy consumed by the system.
            
            \item \underline{Chapter \ref{section: data-extensions}: \emph{Data-efficient training extensions}}. Additional adaptations will be made that reduce the necessary amount of training data, further lowering resource requirements.
            
            \item \underline{Chapter \ref{section: evaluation}: \emph{Extended evaluation}}. An analysis will be made between the baseline and extended models, comparing the performance and efficiency of each, and discussing their success in reducing resource requirements.
        \end{itemize}

        \textbf{\\Contributions to science:} 
        \begin{enumerate}
            \item \emph{Expanding the applications of Green AI}. The first application of Green AI to the finance sector, further demonstrating the utility and importance of Green AI to lowering the environmental cost of deep learning. 

            \item \emph{Reducing the environmental impact of financial technology}. The improvement of sustainable finance to include the new research domain of \emph{sustainable deep learning for sustainable finance}. 

            \item \emph{Improving the inclusivity of finance}. Lowering the bar-to-entry to engage in deep learning for finance,  allowing more individuals to leverage financial technology and analytics.
        \end{enumerate}
        
        \textbf{\\ \\Keywords:} Green AI, Green Deep Learning, Energy Efficiency, Data Efficiency, Sustainable Finance, Financial Volatility Modelling, Long Short-Term Memory

    \end{abstract}


    % -----------------  ACKNOWLEDGEMENTS  -------------------
    \newpage
    \chapter*{Acknowledgements}
    \addcontentsline{toc}{chapter}{Acknowledgement}
    \blindtext 


    % -----------------  TABLE OF CONTENTS -------------------
    \newpage
    \tableofcontents


    % -------------------  LIST OF FIGURES --------------------
    \newpage 
    \listoffigures
    \addcontentsline{toc}{chapter}{List of Figures}


    % -------------------  LIST OF TABLES ---------------------
    \newpage
    \listoftables 
    \addcontentsline{toc}{chapter}{List of Tables}




    % --------------------  INTRODUCTION ----------------------
    \newpage
    \pagenumbering{arabic}
    \chapter{Introduction}
    \label{chapter: intro}

    Test of citations \citep{xu-2021}.

    \section{Topic \& Background}
    \label{section: topic}

    \section{Research Motivations \& Objectives}
    \label{section: motivations}

    \section{Research Methodology}
    \label{section: method}

    \section{Contributions to Science}
    \label{section: contributions}

    \section{Outline of Structure}
    \label{section: structure}


    % --------------------  LITERATURE REVIEW ----------------------
    \newpage
    \chapter{Background \& Literature Review}
    \label{chapter: literature}


    % --------------------  EXPERIMENTS ----------------------
    \newpage
    \chapter{Experiments}
    \label{chapter: experiments}

    \section{Baseline Financial Volatility Model}
    \label{section: baseline}

    \section{Energy-Efficient Training Extensions}
    \label{section: energy-extensions}

    \section{Data-Efficient Training Extensions}
    \label{section: data-extensions}

    \section{Extended evaluation}
    \label{section: evaluation}


    % --------------------  CONCLUSIONS ----------------------
    \newpage
    \chapter{Conclusions \& Future Work}
    \label{chapter: conclusion}


    % --------------------  BIBLIOGRAPHY ---------------------
    \newpage
    \bibliography{bibliography}

\end{document}